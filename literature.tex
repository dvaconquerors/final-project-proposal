\documentclass{article}
\usepackage[utf8]{inputenc, markdown}

\title{Literature DRAFT}
\author{Guenter}
\date{March 2021}

\begin{document}

\maketitle

\begin{markdown}
# Literature that could be added for:
## Real Estate Price Estimation

## Regression in general

[man]**R Manjula et al.** Real estate value prediction using multivariate regression models. IOP Converence Series: Materials Science and Engineering. vol. 263, Issue 4. 2017



## Salary development over time

## Air quality:

[ans]**Luc Anselin & Julie Le Gallo**. Interpolation of Air Quality Measures in Hedonic House Price Models: Spatial Aspects. Spatial Economic Analysis. vol 1. 2006. 31-52, DOI: 10.1080/17421770600661337, https://rsa.tandfonline.com /doi/full/10.1080/17421770600661337 Accessed 3 Mar. 2021.

[bay]**Patrick Bayer, Nathaniel Keohane, Christopher Timmins.** Migration and hedonic valuation: The case of air quality. Journal of Environmental Economics and Management. Volume 58, Issue 1. 2009. pp. 1-14. ISSN 0095-0696. https://doi.org/10.1016/j.jeem.2008.08.004 (https://www.sciencedirect.com/science/article/pii/S0095069609000035) Accessed 4 Mar.

**Fernando Carriazo and John Alexander Gomez-Mahecha**. “The demand for air quality: evidence from the
housing market in Bogotá, Colombia” Environment and Development Economics, no. 23, 2018, pp. 121-138,  https://doi.org/10.1017/S1355770X18000050. Accessed 3 Mar. 2021.

[ceh]**Čeh, M.; Kilibarda, M.; Lisec, A.; Bajat, B.** Estimating the Performance of Random Forest versus Multiple Regression for Predicting Prices of the Apartments. ISPRS Int. J. Geo-Inf. 2018, 7, 168. https://doi.org/10.3390/ijgi7050168. Accessed 4 Mar 2021.


[kim]**Chong Won Kim, Tim T Phipps, Luc Anselin**. Measuring the benefits of air quality improvement: a spatial hedonic approach. Journal of Environmental Economics and Management. vol. 45, Issue 1. 2003. pp 24-39. https://doi.org/10.1016/S0095-0696(02)00013-X. (Accessible through: https://ageconsearch.umn.edu/record/20959/files/spkimc01.pdf) Accessed 3 Mar 2021.

**Liu, Runqiu et al**. “Impacts of Haze on Housing Prices: An Empirical Analysis Based on Data from Chengdu (China).” International journal of environmental research and public health vol. 15,6 1161. 2 Jun. 2018. doi:10.3390/ijerph15061161. https://www.ncbi.nlm.nih.gov/pmc/articles/PMC6025591/. Accessed 3 Mar. 2021.

[nes]**Nesticò, A.; La Marca, M.** Urban Real Estate Values and Ecosystem Disservices: An Estimate Model Based on Regression Analysis. Sustainability 2020, 12, 6304. https://doi.org/10.3390/su12166304 (Accessible through: https://www.mdpi.com/2071-1050/12/16/6304#cite) Accessed 4 Mar. 2021.

**Daniel M. Sullivan**. "The True Cost of Air Pollution: Evidence from House Prices and Migration". Harvad Environmental Economics Program. Discussion Paper 16-69. May 2016. 
https://heep.hks.harvard.edu/files/heep/files/dp69_sullivan.pdf. Accessed 3 Mar. 2021.

**Zabel, Jeffrey E., and Katherine A. Kiel**. “Estimating the Demand for Air Quality in Four U.S. Cities.” Land Economics, vol. 76, no. 2, 2000, pp. 174–194. JSTOR, www.jstor.org/stable/3147223. Accessed 3 Mar. 2021.

\end{markdown}

\end{document}
